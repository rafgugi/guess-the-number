\vspace{0ex}
\chapter {PENDAHULUAN}
Pada bab ini, akan dijelaskan mengenai latar belakang, rumusan masalah, Batasan masalah, tujuan, metodologi pengerjaan, dan sistematika penulisan tesis.

\section{Latar Belakang}
\par Dalam perkembangan dunia teknologi informasi selama beberapa dekade terakhir, teknologi informasi seringkali dijadikan solusi bagi permasalahan-permasalahan yang pernah ada, yang sebelumnya diselesaikan secara manual oleh manusia. Contoh permasalahan yang pernah ada adalah salah satu permasalahan klasik pencarian Ulam dan Rényi. Permasalahan ini dapat diilustrasikan dengan adanya dua pemain yang disebut penanya dan penjawab. Diberikan range pertanyaan $S_M = {0,\ldots,M-1}$, penjawab menentukan sebuah bilangan $x \in S_M$. Penanya harus menemukan nilai $x$ dengan memberikan beberapa query berupa pertanyaan iya dan tidak, apakah "$x \in Q$?", di mana $Q$ adalah subset dari $S_M$, lalu penjawab menjawab "ya" atau "tidak". Permasalahan utama adalah penjawab dapat berbohong sampai $e$ kali. Masalah dari permainan ini adalah mencari jumlah query minimal untuk dapat menentukan nilai $x$.

Pada permasalahan pencarian Ulam, penanya dan penjawab harus menyepakati beberapa peraturan sebelum bermain. Peraturan tersebut meliputi batasan ruang pencarian, batasan bagaimana penjawab diperbolehkan berbohong, format pertanyaan, dan bagaimana interaksi antara penjawab dan penanya \cite{Pelc2002}. Pertama penanya dan penjawab harus menyepakati batas ruang pencarian $S_M$, yaitu $M$ angka, penjawab hanya boleh menentukan angka $x$ diantara dalam set $\{0,\ldots,M-1\}$.

Aturan bagaimana penjawab diperbolehkan berbohong adalah aturan yang fundamental dalam permainan pencarian Ulam dan Rényi. Aturan probabilitas berbohong dicetuskan oleh Rényi dan aturan jumlah bohong dicetuskan oleh Ulam \cite{StanislawMUlam1976}. Pada kebohongan probabilitas yang diikiat secara global, probabilitas penjawab melakukan kebohongan ditentukan oleh $r$, sehingga maksimal penjawab melakukan kebohongan adalah $rn$ di mana $n$ adalah jumlah pertanyaan dan $r<1$ \cite{Dhagat1999}. Pada aturan jumlah bohong, variasi beragam antara maksimal penjawab dapat berbohong hanya satu \cite{Ellis2008} \cite{Pelc1988}, dua \cite{Cicalese2000}, tiga \cite{Negro1992}, dan lebih dari tiga \cite{Berlekamp1998} \cite{Deppe2004}.

Pada aturan format pertanyaan, terdapat beberapa variasi. Yang pertama adalah pertanyaan komparasi, bentuk pertanyaannya adalah "Apakah $x<a$?" di mana $a \in S_M$ \cite{Innes} \cite{Auletta1992}. Lalu ada pertanyaan interval dan bi-interval, bentuk pertanyaannya adalah "Apakah $x$ ada dalam interval $[a,b]$?" \cite{Peter2017} dan "Apakah $x$ ada dalam interval $[a,b] \cup [c,d]$?" \cite{Mundici1997}. Lalu format pertanyaan subset, bentuk pertanyaannya adalah "Apakah $x$ ada dalam $A$ di mana $A \subseteq S_M$" \cite{Katona} \cite{Macula1997}.

Pada aturan interaksi antara penjawab dan penanya, terdapat tiga macam variasi yaitu interaktif, batch, dan non interaktif. Aturan yang paling umum digunakan adalah interaktif, yaitu penjawab harus menjawab setiap pertanyaan yang diajukan penanya sebelum penanya menanyakan pertanyaan selanjutnya. Pada aturan batch, penanya dan penjawab menyepakati berapa jumlah batch. Lalu pada setiap batch, penanya memberikan beberapa pertanyaan, lalu penjawab memberikan jawaban sejumlah pertanyaan yang diberikan oleh penanya \cite{Cicalese2000}. Aturan yang terakhir adalah non interaktif, yaitu penjawab harus menjawab semua pertanyaan penanya sekaligus \cite{Macula1997}.

Salah satu variasi permasalahan Ulam dan Rényi yang diangkat dalam penelitian ini adalah pencarian Ulam dengan $n$ query subset ${q_1,q_2,\ldots,q_n} | q_i \in S_M$, maksimal bohong adalah $e$, dan penjawab hanya boleh menjawab query penanya setelah penanya selesai menanyakan semua query-nya. Belum ada penelitian yang menyelesaikan permasalahan ini. Oleh karena itu penelitian ini bertujuan untuk memberikan solusi pada permasalahan ini.


\section {Rumusan Masalah}
Permasalahan yang akan diselesaikan pada tesis ini adalah sebagai berikut:

\begin {enumerate}
  \item Bagaimana merumuskan query untuk mencari bilangan diskrit pada interval yang diberikan pada permasalahan pencarian Ulam dengan pertanyaan seminimal mungkin?
  \item Apakah algoritma kode biner dengan jarak Hamming dapat menyelesaikan permasalahan pencarian Ulam non interaktif lebih baik?
  \item Bagaimana implementasi struktur data yang efisien dan optimal untuk menyelesaikan permasalahan pencarian Ulam?
  \item Apakah solusi meggunakan algoritma kode biner dengan tabel pencarian dapat diterima pada pengujian online SPOJ?
\end {enumerate}


\section {Batasan Masalah}
Masalah yang akan diselesaikan memiliki batasan-batasan berikut:

\begin {enumerate}
  \item Implementasi algoritma menggunakan bahasa pemrograman C++.
  \item Batas maksimum kasus uji adalah $2^7$.
  \item Pada setiap permainan, penjawab dan penanya menyepakati jumlah $M$ dan $e$.
  \item Interval ruang pencarian adalah $[1,M]$, dengan $M$ maksimum $2^{12}$.
  \item Jumlah maksimal penjawab dapat berbohong adalah $2 \leq e \leq 16$.
  \item Dataset yang digunakan adalah dataset pada permasalahan SPOJ GUESSN5.
\end {enumerate}


\section {Tujuan}
Tujuan tesis ini adalah sebagai berikut:

\begin{enumerate}
  \item Melakukan analisis dan mendesain algoritma dan struktur data untuk mencari bilangan dengan kebohongan dalam studi kasus permasalahan pencarian Ulam non interaktif dengan kebohongan.
  \item Mengevaluasi hasil dan kinerja algoritma dan struktur data yang telah dirancang untuk permasalahan pencarian Ulam non interaktif dengan kebohongan.
\end{enumerate}


\section {Sistematika Penulisan}
Sistematika laporan tesis yang akan digunakan adalah sebagai berikut:

\begin{enumerate}
\item Bagian awal, meliputi halaman depan, halaman pengesahan, abstrak, kata pengantar, daftar isi, daftar gambar, dan daftar tabel.
\item Bagian inti, meliputi pendahuluan, tinjauan pustaka, metodologi, hasil dan pembahasan, dan kesimpulan dan saran.
\item Bagian akhir, meliputi daftar pustaka, lampiran-lampiran, dan biodata penulis.
\end{enumerate}