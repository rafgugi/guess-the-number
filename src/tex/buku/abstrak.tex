% ---- Indonesian vers.
\chapter {ABSTRAK}
\noindent\textbf{\MakeUppercase\judul}
\vspace*{1em}

\begin{tabularx}{\linewidth}{ l l X }
  Nama       & : & \penulis \\
  NRP       & :  & \nrplama \\
  % Departemen     & : & \jurusanbaru, \newline \fakultasbaru, ITS \\
  Pembimbing     & : & \pembimbingsatu
  \vspace*{1em}   % HACKY--USE ALTERNATIVE IF POSSIBLE %
\end{tabularx}

\noindent\textbf{\large Abstrak} \\
\itshape
Pada permasalahan permainan klasik pencarian Ulam dan Rényi, penanya harus mengajukan beberapa pertanyaan iya dan tidak untuk mencari sebuah nilai dalam range pencarian yang sudah disepakati, namun penjawab diperbolehkan berbohong. Sudah ada solusi dari beberapa variasi pada permasalahan pencarian Ulam dan Rényi, yaitu pada jenis query antara rentang atau subset dan jumlah maksimal bohong antara satu, dua, tiga, dan seterusnya. Namun belum ada solusi yang sempurna untuk query yang non-interaktif yaitu penjawab hanya boleh menjawab query penanya setelah penanya selesai menanyakan semua querynya. Pada paper ini akan dijelaskan solusi sempurna untuk permainan Ulam dan Rényi non-interaktif dengan maksimal kebohongan jamak menggunakan kode biner dengan jarak Hamming.

\vspace*{1em}
\noindent\bfseries Kata Kunci: permainan ulam; bohong; jarak hamming; kode biner; query;
\normalfont
\cleardoublepage

% ---- English vers.
\chapter {ABSTRACT}
\noindent\textbf{\MakeUppercase\juduleng}
\vspace*{1em}

\begin{tabularx}{\linewidth}{ l l X }
  Name       & : & \penulis \\
  Student ID    & :  & \nrplama \\
  % Department     & : & \jurusanbarueng, \newline \fakultasbarueng, ITS \\
  Supervisor    & : & \pembimbingsatu
  \vspace*{1em}   % HACKY--USE ALTERNATIVE IF POSSIBLE %
\end {tabularx}

\noindent\textbf{\large Abstract} \\
\itshape
On the classic Ulam and Rényi searching problem, the questioner has to ask some yes-no questions to find an unknown value within the agreed search space, but the answerer is allowed to lie. There are already solutions of some variations on the Ulam and Rényi searching problem, ie on the type of query between range or subset and the maximum number of lies between one, two, three, and so on. But there is no perfect solution for non-interactive queries which the answerer can only answer the questioner's query after the questioner has finished querying all the queries. In this paper we will describe the perfect solution for non-interactive Ulam and Rényi searching problem with many lies using binary code with Hamming distance.

\vspace*{1em}
\noindent\bfseries Keywords: ulam game; lies; hamming distances; binary codes; query;
\normalfont
\cleardoublepage