\chapter{KESIMPULAN DAN SARAN}

Bab ini membahas mengenai kesimpulan dan saran yang dapat diambil dari hasil uji coba yang telah dilakukan sebagai jawaban dari rumusan masalah yang dikemukakan.

\section{Kesimpulan}

% Kesimpulan yang dapat diambil adalah algoritma kode biner menunjukkan hasil yang jauh lebih bagus dilihat dari jumlah query yang dihasilkan, namun penggunaan memori dan waktu yang cukup banyak. Oleh karena itu algoritma kode biner dapat dioptimasi dengan tabel pencarian sehingga dapat mengurangi waktu eksekusi dan penggunaan memori sampai mendekati performa algoritma repetisi biner.

Berdasarkan penjabaran di bab-bab sebelumnya, dapat disimpulkan beberapa poin terkait penyelesaian permasalahan Ulam non interaktif.

\begin{enumerate}
	\item Permasalahan Ulam non interaktif dapat diselesaikan dengan algoritma repetisi biner maupun kode biner dengan jarak Hamming.
	\item Algoritma kode biner dengan jarak Hamming memiliki hasil yang lebih baik dibandingkan algoritma repetisi biner dilihat dari sisi jumlah query.
	\item Pada algoritma kode biner terdapat proses pencarian mendalam yang memiliki kompleksitas $O(M^3)$, namun dapat direduksi menjadi $O(1)$ dengan melakukan proses pencarian mendalam sebagai pra proses dan menghasilkan waktu dan penggunaan memori yang sama bagusnya dengan algoritma repetisi biner.
	\item Solusi meggunakan algoritma kode biner dengan tabel pencarian dapat diterima pada pengujian online SPOJ dan menghasilkan peringkat paling tinggi dibandingkan solusi yang sudah diselesaikan peserta lain.
\end{enumerate}

\section{Saran}

Algoritma solusi menggunakan kode biner dengan pra proses pencarian mendalam terkendala dengan pra proses yang harus dilakukan dengan batasan jumlah bohong dan ruang pencarian yang harus ditentukan sejak awal. Saran untuk penelitian terkait adalah bagaimana solusi dapat dilakukan secara dinamis tanpa harus menambahkan kode yang didapat dari pra proses.
