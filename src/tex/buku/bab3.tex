\chapter{METODE PENELITIAN}

Bab ini memaparkan tentang metodologi penelitian yang digunakan pada penelitian ini.

\section{Solusi menggunakan repetisi kode biner}

Pada permasalahan pencarian Ulam non-interaktif, penjawab tidak diperbolehkan menjawab sebelum penanya selesai menanyakan seluruh query. Pendekatan pertama yang mungkin untuk menyelesaikan permasalahan ini adalah dengan mempersiapkan pencarian biner. Query awal pencarian biner berjumlah $q_b=ceil(log_2(n))$, yaitu bilangan bulat minimal yang jika dipangkat dua akan bernilai minimal $n$, agar setiap kemungkinan jawaban dari penjawab dapat mewakili semua kemungkinan nilai $x$. Lalu setiap query akan diulang sebanyak $q_e=2e+1$ kali agar penjawab pasti menjawab dengan jujur, karena $e$ query untuk jawaban bohong, ditambah dengan $e$ query untuk mengeliminasi jawaban bohong, ditambah dengan satu query jawaban pasti jujur karena kesempatan penjawab untuk berbohong sudah habis. Total jumlah query $q=q_b \times q_e$.

\begin{table}[h!]
\caption{Contoh pencarian biner pada $n=8$}
\label{tab:binary_8}
\begin{center}
\begin{tabu} {|X[l]|X[c]|X[c]|X[c]|X[c]|X[c]|X[c]|X[c]|X[c]|} 
\hline
$x$  & 1 & 2 & 3 & 4 & 5 & 6 & 7 & 8 \\
\hline
$Q_1$ & 0 & 0 & 0 & 0 & 1 & 1 & 1 & 1 \\
\hline
$Q_2$ & 0 & 0 & 1 & 1 & 0 & 0 & 1 & 1 \\
\hline
$Q_3$ & 0 & 1 & 0 & 1 & 0 & 1 & 0 & 1 \\
\hline
Jawaban & NNN & NNY & NYN & NYY & YNN & YNY & YYN & YYY \\
\hline
\end{tabu}
\end{center}
\end{table}

\section{Solusi menggunakan pembobotan Berlekamp}

Query pencarian biner dengan repetisi pasti dapat menyelesaikan permasalahan Ulam non-interaktif, namun jumlah query terlalu banyak dan redundansi. Oleh karena itu diperlukan pendekatan statistik peluang menggunakan pembobotan berlekamp untuk mereduksi jumlah query. Gambar \ref{fig:flow_berlekamp} menjelaskan alur proses adalah membuat query pencarian biner terlebih dahulu, lalu dioptimasi menggunakan fungsi pembobotan berlekamp untuk menghitung peluang dari semua angka-angka dalam range $S_n$.

\begin{figure}
\centering
\includegraphics[scale=0.4]{../img/flowchart-berlekamp.png}
\caption{Diagram alir solusi menggunakan pembobotan Berlekamp}
\label{fig:flow_berlekamp}
\end{figure}

Misalnya jika $n=8$, $e=2$, $m=6$, maka jumlah $q_b$ untuk pencarian biner adalah 00001111, 00110011 dan 01010101 seperti pada Tabel \ref{tab:binary_8}. Dari tiga query, tersebut, semua jawaban penjawab mulai dari "NNN" sampai "YYY" dapat mewakili semua nilai $x$ dalam $S_n={1,2,...,8}$ sehingga nilai $q_b=3$. Lalu masing-masing query diulang sebanyak $q_e=2e+1=5$ kali. Maka total dari $q=q_b \times q_e=9$.

\begin{figure}
\centering
\begin{BVerbatim}
1001101
0101011
0010111
\end{BVerbatim}
\caption{Generator matrix $[7,3,4]_2$}
\label{fig:generator734}
\end{figure}

\begin{figure}
\centering
\begin{BVerbatim}
0000000  1000111
0011101  1000111
0101011  1000111
0110110  1000111
\end{BVerbatim}
\caption{Perfect binary code $(7,8,4)_2$}
\label{fig:binarycode784}
\end{figure}

Now again see a generator matrix $[7,3,4]_2$ in Figure \ref{fig:generator734} can create a perfect binary code $(7,8,4)_2$ code in Figure \ref{fig:binarycode784}. A generator matrix $[n, m, d]_2$ where $n = 2^m - 1$ and each \textbf{column} is a linear combination of $m$ bit binary string except $\vec{0}$ can make a perfect binary code $(n,M,d)_2$ where $d = M/2$. \textbf{\textit{We need a profing or citation..}}

Nah masalahnya, jika $d$ yang kita butuhkan kurang dari $M/2$, maka kita harus menggunakan seminimal mungkin $n$ kolom pertama $[M-1,M,M/2]_2$ sehingga menghasilkan $[n,M,d]_2$. Asumsikan kita memiliki fungsi $\aleph(d)$ adalah berapa kolom pada generator matrix yang akan menghasilkan kode biner dengan minimal jarak Hamming $d$. Hal yang pasti adalah $\aleph(M/2) = M-1$ dan $\aleph(1) = {log}_2(m)$ (need proving). Worst case adalah $\aleph(d)=M/2 + d - 1$. Best case adalah karena ada $M-1$ kolom untuk $0 \leq d \leq M/2$, maka $\aleph(d) = d/2$ (sepertinya tidak mungkin). Nah problem kita adalah bagaimana mengurutkan kolom pada generator matrix sehingga $\aleph(d)$ sesempurna mungkin, atau bisa dikatakan $\aleph(d) - \aleph(d-1) \approx 2$.
