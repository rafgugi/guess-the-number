\chapter{METODE PENELITIAN}

Bab ini memaparkan tentang metodologi penelitian yang digunakan pada penelitian ini.

\section{Analisis penyelesaian masalah}

Pada permasalahan pencarian Ulam non-interaktif, penjawab tidak diperbolehkan menjawab sebelum penanya selesai menanyakan seluruh query. Ada beberapa pendekatan yang dapat dilakukan untuk menyelesaikan masalah tersebut.

\subsection{Solusi menggunakan repetisi pencarian biner}

Pendekatan pertama yang mungkin untuk menyelesaikan permasalahan ini adalah dengan mempersiapkan pencarian biner. Query awal pencarian biner berjumlah $q_b=\log_2(n)$, agar setiap kemungkinan jawaban dari penjawab dapat mewakili semua kemungkinan nilai $x$. Lalu setiap query akan diulang sebanyak $q_e=2e+1$ kali agar penjawab pasti menjawab dengan jujur, karena $e$ query untuk jawaban bohong, ditambah dengan $e$ query untuk mengeliminasi jawaban bohong, ditambah dengan satu query jawaban pasti jujur karena kesempatan penjawab untuk berbohong sudah habis. Total jumlah query $q=q_b \times q_e$.

\begin{table}[h!]
\caption{Contoh pencarian biner pada $n=8$}
\label{tab:binary_8}
\begin{center}
\begin{tabu} {|X[l]|X[c]|X[c]|X[c]|X[c]|X[c]|X[c]|X[c]|X[c]|}
\hline
$x$  & 1 & 2 & 3 & 4 & 5 & 6 & 7 & 8 \\
\hline
$Q_1$ & 0 & 0 & 0 & 0 & 1 & 1 & 1 & 1 \\
\hline
$Q_2$ & 0 & 0 & 1 & 1 & 0 & 0 & 1 & 1 \\
\hline
$Q_3$ & 0 & 1 & 0 & 1 & 0 & 1 & 0 & 1 \\
\hline
Jawaban & NNN & NNY & NYN & NYY & YNN & YNY & YYN & YYY \\
\hline
\end{tabu}
\end{center}
\end{table}

Misalnya jika $n=8$, $e=2$, $m=6$, maka jumlah $q_b$ untuk pencarian biner adalah \texttt{00001111}, \texttt{00110011} dan \texttt{01010101} seperti pada Tabel \ref{tab:binary_8}. Dari tiga query, tersebut, semua jawaban penjawab mulai dari "NNN" sampai "YYY" dapat mewakili semua nilai $x$ dalam $S_n={1,2,...,8}$ sehingga nilai $q_b=3$. Lalu masing-masing query diulang sebanyak $q_e=2e+1=5$ kali. Maka total dari $q=q_b \times q_e=9$.


\subsection{Solusi menggunakan kode biner}

Tujuan dari permasalahan ini adalah untuk menghasilkan $n$ query. Diberikan sebuah matriks $L$ berukuran $n \times M$ berisi $n$ query. Kumpulan query ini dinotasikan dengan $L = \{\vec{q_1},\vec{q_2},\ldots,\vec{q_n}\}$ di mana $\vec{q_i} = \{s_1,s_2,\ldots,s_M\}$. Himpunan nilai $s_i$ yang mungkin adalah $s_i \in \mathbb{F}_2$. Diberikan sebuah vektor $\vec{z} \in \{z_1,z_2,\ldots,z_n\}$ di mana $z_i \in \{0,1\}$ berisi jawaban dari seluruh query secara berurutan, $z_i$ adalah jawaban dari $\vec{q_i}$, di mana $0$ berarti 'tidak' dan $1$ berarti 'ya'. Karena jika jawaban $0$ berarti query harus ditambah dengan 1 dan jika jawaban $1$ berarti query ditambah dengan 0 (diabaikan), maka kita memiliki $\vec{z'}$ yaitu inverse dari $\vec{z}$.

\begin{lemma}
Diketahui integer $n$, $M$, dan $d$. Jika $L'$ adalah kode biner $(n,M,d)_2$ yang valid, maka jika setiap codeword $\vec{c}$ pada $L'$ ditambah dengan $\vec{z} \mid \vec{z} \in \mathbb{F}_2^n$ maka hasilnya akan tetap menjadi kode biner $(n,M,d)_2$ yang valid.
\end{lemma}

\begin{proof}
Jika $d_H(\vec{x_i},\vec{y_j}) \ge d \mid \vec{x_i},\vec{y_j} \in L'$ maka
\begin{align*}
d_H(\vec{x_i}+\vec{z},\vec{y_j}+\vec{z}) &\ge d \\
wt(\vec{x_i}+\vec{y_j}+\vec{z}+\vec{z}) &\ge d \\
wt(\vec{x_i}+\vec{y_j}) &\ge d \\
d_H(\vec{x_i},\vec{y_j}) &\ge d \\
\end{align*}
Jadi jika $d_H(\vec{x_i},\vec{y_j}) \ge d$ maka $d_H(\vec{x_i}+\vec{z},\vec{y_j}+\vec{z}) \ge d$.
\end{proof}

Matriks $L'$ berukuran $M \times n$ adalah hasil transpose dari matriks $L$. Tambahkan seluruh baris pada $L'$ dengan $z'$. Maka jawaban dari permainan Ulam non-interaktif adalah index dari baris $r$ pada $L'$ yang memiliki bobot $wt(\vec{x_r}) > n-e$.

Penanya memenangkan permainan jika $L'$ memiliki paling banyak satu row dengan $wt(\vec{x}) \ge n-e$. Jika hanya ada satu row, maka row tersebut adalah jawaban permainan. Jika tidak ada satu row pun yang memenuhi, penanya tetap memenangkan permainan karena penjawab melakukan kecurangan, melakukan bohong untuk semua angka lebih dari batas yang ditetapkan.

Untuk meyakinkan bahwa setelah seluruh jawaban $\vec{z}$ diberikan dan diaplikasikan ke matrix $L$ dan tidak pasti hanya ada 1 baris yang memiliki nilai $1$ antara $n-e \le wt(\vec{x_r}) \le n$, adalah dengan memastikan bahwa jarak Hamming setiap row yang berbeda pada $L'$ adalah minimal $d$.

\begin{lemma}
Diketahui integer $n$, $M$, dan $d$. Jika $L'$ adalah kode biner $(n,M,d)_2$ yang valid, maka pasti hanya ada paling banyak satu codeword $\vec{c}$ yang memiliki $0 \le wt(\vec{c}) \le e$.
\end{lemma}

\begin{proof}
Jika ada codeword $c$ di mana $0 \le wt(c) \le e$ maka $wt(\vec{x}) > e$ di mana $\vec{x} \neq c$. Pembuktian dapat dibuktikan dengan dua kasus.\\

1) Jika $wt(\vec{c}) = 0$ maka
\begin{align*}
d_H(\vec{c},\vec{x}) &\ge d \mid \vec{c} \neq \vec{x} , \vec{x} \in L' \\
wt(\vec{c} + \vec{x}) &\ge d \\
wt(\vec{x}) &\ge d \label{eq:proofd} \stepcounter{equation} \tag{\theequation}
\end{align*}
Sebelumnya telah disebutkan hubungan $d$ dan $e$ pada Persamaan \ref{eq:de}. Persamaan tersebut dapat diturunkan menjadi
\begin{equation} \label{eq:dge}
d > e
\end{equation}
Dengan memasukkan Persamaan \ref{eq:dge} ke Persamaan \ref{eq:proofd}, maka didapatkan
\begin{equation*}
wt(\vec{x}) > e
\end{equation*}

2) Jika $wt(\vec{c}) = e$ maka
\begin{align*}
d_H(\vec{c},\vec{x}) &\ge d \mid \vec{c} \neq \vec{x} , \vec{x} \in L' \\
wt(\vec{c}+\vec{x}) &\ge d \\
wt(\vec{c}) + wt(\vec{x}) \ge wt(\vec{c}+\vec{x}) &\ge d \\
wt(\vec{c}) + wt(\vec{x}) &\ge d \\
e + wt(\vec{x}) &\ge d \\
\intertext{Masukkan Persamaan \ref{eq:de} untuk mensubstitusi $d$}\\
wt(\vec{x}) &\ge 2e+1-e \\
&\ge e+1\\
wt(\vec{x}) &> e
\end{align*}
Jadi jika ada codeword $c$ di mana $0 \le wt(c) \le e$ maka $wt(\vec{x}) > e$ di mana $\vec{x} \neq c$.
\end{proof}

Dari pembuktian diatas, dapat disimpulkan bahwa untuk menyelesaikan permainan pencarian Ulam non-interaktif dengan batas pencarian $M$ dan maksimal kebohongan $e$, transpose dari $n$ query yang dibuat harus membentuk kode biner $(n,M,d)_2$.


\section{Implementasi algoritma}

Implementasi merupakan tahap untuk membangun algoritma yang akan digunakan. Pada tahap ini dilakukan implementasi dari rancangan struktur data yang akan dimodelkan sesuai dengan permasalahan. Implementasi dilakukan dengan menggunakan bahasa pemrograman C++ agar dapat disubmit ke SPOJ.

Implementasi dalam bahasa coffeescript untuk menghasilkan halaman web. Visualisasi dalam bentuk web digunakan untuk mencari pola pada query karena penampilan yang lebih dapat dipahami. Selain itu akan dibuat visualisasi untuk memudahkan pembacaan jarak Hamming pada kode biner. Aplikasi web akan menerima input, lalu menampilkan visualisasi sesuai dengan kebutuhan.


\section{Pengujian}

Tahap pengujian adalah melakukan uji coba menggunakan dataset pada Online Judge SPOJ GUESSN5 untuk mengetahui hasil dan performa dari metode dan struktur data yang dibangun. Hal yang dinilai pada pengujian adalah skor, penggunaan memory, dan waktu yang dibutuhkan. Pembobotan skor adalah jika penjawab menemukan ada suatu set jawaban yang menyebabkan lebih dari satu kemungkinan nilai $x$, maka pengujian dianggap gagal. Jika berhasil, maka nilai skor bertambah $q^2$. Jika gagal, maka nilai skor bertambah $4m^2$. Total skor adalah jumlah semua skor dari setiap kasus uji. Semakin besar skor maka semakin baik algoritma yang digunakan.

Evaluasi dan perbaikan juga akan dilakukan pada Online Judge hingga perangkat lunak yang diuji mengeluarkan hasil dan performa yang sesuai dengan data uji pada Online Judge SPOJ.