\chapter{HASIL DAN PEMBAHASAN}

\section{Lingkungan Uji Coba}

Uji coba dilakukan pada perangkat dengan spesifikasi berikut.
\begin{enumerate}
  \item Perangkat Keras
  \begin{enumerate}
    \item Processor Intel® Core™ i5-7400 CPU @ 3.00GHz (4 CPUs), ~3.0GHz
    \item Random Access Memory 8192MB
  \end{enumerate}
  \item Perangkat Lunak
  \begin{enumerate}
    \item Sistem Operasi Linux Ubuntu 16.04
    \item Bahasa Pemrograman C++
    \item gcc 5.4.0
  \end{enumerate}
\end{enumerate}

\section{Implementasi Program Utama}

Subbab ini menjelaskan implementasi program utama. Program ini merupakan program yang digunakan untuk menyelesaikan permasalahan DSA Attack.

\subsection{Penggunaan Library, Tipe data, dan Struct}

Program ini menggunakan beberapa \textit{library} dan tipe data seperti yang ditunjukkan pada kode sumber \ref{code:header_main}.

\begin{code}[firstnumber=1,float]{\textit{Header} Program Utama}{header_main}
#include <cmath>
#include <cstdio>
#include <iostream>
#include <string>
#include <vector>
\end{code}

\section{Skenario Uji Coba}

Subbab ini akan menjelaskan hasil pengujian program penyelesaian permasalahan.