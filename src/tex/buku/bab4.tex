\chapter{HASIL DAN PEMBAHASAN}

Bab ini menjelaskan uji coba yang dilakukan pada sistem yang telah dibangun beserta analisis dari uji coba yang telah dilakukan. Pembahasan pengujian meliputi lingkungan uji coba, skenario uji coba serta analisis setiap pengujian.

\section{Lingkungan Uji Coba}

Uji coba dilakukan pada perangkat dengan spesifikasi berikut.

\begin{enumerate}
  \item Perangkat Keras
  \begin{enumerate}
    \item Processor Intel® Core™ i7-7400 CPU @ 3.00GHz (4 CPUs), ~3.0GHz
    \item Random Access Memory 8192MB
  \end{enumerate}
  \item Perangkat Lunak
  \begin{enumerate}
    \item Sistem Operasi Linux Ubuntu 16.04
    \item Bahasa Pemrograman C++
    \item gcc 5.4.0
  \end{enumerate}
\end{enumerate}

\section{Uji coba menggunakan dataset lokal}

\section{Uji coba menggunakan dataset online SPOJ}

Subbab ini akan menjelaskan hasil pengujian program penyelesaian permasalahan menggunakan algoritma repetisi pencarian biner dan kode biner.

\subsection{Uji coba algoritma repetisi pencarian biner pada SPOJ}

Algoritma repetisi pencarian biner disubmit ke SPOJ dalam bahasa C, menghasilkan penilaian yang ditunjukkan pada Tabel \ref{tab:score_repetitive}. Karena ini adalah algoritma yang pasti benar dengan cara termudah, maka dapat diasumsikan bahwa skor yang didapat dari algoritma ini adalah skor minimal yang dapat menjadi tolok ukur keberhasilan algoritma yang lain.

\begin{table}[h!]
\caption{Hasil algoritma repetisi pencarian biner pada SPOJ}
\label{tab:score_repetitive}
\begin{center}
\begin{tabular} {|l|l|}
\hline
ID & 20152331 \\ \hline
Tanggal & 2017-09-14 06:08:19 \\ \hline
Skor & 42,787,090 \\ \hline
Waktu & 0.00 \\ \hline
Memori & 2.7M \\ \hline
\end{tabular}
\end{center}
\end{table}

\subsection{Uji coba algoritma kode biner pada SPOJ}

Algoritma repetisi pencarian biner disubmit ke SPOJ dalam bahasa C, menghasilkan penilaian yang ditunjukkan pada Tabel \ref{tab:score_binary_code}.

\begin{table}[h!]
\caption{Hasil algoritma kode biner pada SPOJ}
\label{tab:score_binary_code}
\begin{center}
\begin{tabular} {|l|l|}
\hline
ID & 21732463 \\ \hline
Tanggal & 2018-05-27 00:45:01 \\ \hline
Skor & 4,225,555 \\ \hline
Waktu & 0.44 \\ \hline
Memori & 2.8M \\ \hline
\end{tabular}
\end{center}
\end{table}