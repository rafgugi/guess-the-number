\chapter{HASIL DAN PEMBAHASAN}

Bab ini memaparkan tentang hasil dan pembahasan pada metode-metode yang digunakan pada penelitian ini.

\section{Lingkungan Uji Coba}

Uji coba dilakukan pada perangkat dengan spesifikasi berikut.

\begin{enumerate}
  \item Perangkat Keras
  \begin{enumerate}
    \item Processor Intel® Core™ i7-7400 CPU @ 3.00GHz (4 CPUs), ~3.0GHz
    \item Random Access Memory 8192MB
  \end{enumerate}
  \item Perangkat Lunak
  \begin{enumerate}
    \item Sistem Operasi Linux Ubuntu 16.04
    \item Bahasa Pemrograman C++
    \item gcc 5.4.0
  \end{enumerate}
\end{enumerate}


\section{Penyelesaian menggunakan repetisi pencarian biner}

Algoritma repetisi pencarian biner menghasilkan maksimal sebanyak $\log_2(M) \cdot (2e + 1)$ query non interaktif pada setiap kasus uji. Algoritma pencarian biner ada pada Kode Sumber \ref{alg:repetisi_biner}. Baris \ref{alg:for_qb} menunjukkan perulangan untuk setiap jenis query $qb$. Nilai $qb$ adalah hasil $\log_2$ dari M, lalu dibulatkan ke atas karena $M$ harus merupakan perpangkatan dari 2 sehingga nilai $qb = \text{ceil}(\log_2(M))$. Baris \ref{alg:for_m} menunjukkan perulangan untuk membuat setiap satu jenis query pencarian biner. Baris \ref{alg:find_bit} menunjukkan proses untuk mencari bit pada posisi tertentu pada sebuah integer \cite{bithack}. Setiap query akan diulang sebanyak $qe$ seperti yang ditunjukkan pada baris \ref{alg:for_qe}.

\begin{algorithm}[h]
\caption{Algoritma repetisi pencarian biner}
\label{alg:repetisi_biner}
\Fn{repetitive\_binary\_search($M$, $e$)}{
\KwData{$M$ search space, $e$ max lies allowed}
\KwResult{$queries$}
  $queries = [\ ]$\;
  $qb = \text{ceil}(\log_2(M))$\;
  $qe = 2*e + 1$\;
  $twopower = 1$\;
  \For{$i = 0$ \KwTo $qb$}{\label{alg:for_qb}
    $string = ""$\;
    \For{$j = 0$ \KwTo $M$}{\label{alg:for_m}
      \eIf{$j\ \&\ twopower$}{\label{alg:find_bit}
        $string \mathrel{+}= "1"$\;
      }{
        $string \mathrel{+}= "0"$\;
      }
    }
    \For{$j = 0$ \KwTo $qe$}{\label{alg:for_qe}
      $queries$.push($string$)\;
    }
    $twopower \mathrel{*}= 2$\;
  }
  \Return $queries$\;
}
\end{algorithm}

\subsection{Skenario Uji Coba}

Subbab ini akan menjelaskan hasil pengujian program penyelesaian permasalahan. Algoritma disubmit ke SPOJ dalam bahasa C, menghasilkan penilaian yang ditunjukkan pada Tabel \ref{tab:score_repetitive}. Karena ini adalah algoritma yang pasti benar dengan cara termudah, maka dapat diasumsikan bahwa skor yang didapat dari algoritma ini adalah skor minimal yang dapat menjadi tolok ukur keberhasilan algoritma yang lain.

\begin{table}[h!]
\caption{Hasil algoritma repetisi pencarian biner pada SPOJ}
\label{tab:score_repetitive}
\begin{center}
\begin{tabular} {|l|l|}
\hline
ID & 20152331 \\ \hline
Tanggal & 2017-09-14 06:08:19 \\ \hline
Skor & 42,787,090 \\ \hline
Waktu & 0.00 \\ \hline
Memori & 2.7M \\ \hline
\end{tabular}
\end{center}
\end{table}

\section{Penyelesaian menggunakan kode biner}
