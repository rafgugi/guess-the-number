%-------------------------------------------------------------
% utils.tex
% Generic commands which may be used throughout the document
% should be set here
%-------------------------------------------------------------

%--
%  A set of command for appendix testing tables
%--
\newcommand {\testtableheader}
{
  \hline
  No & \multicolumn{3}{|c|}{BSGS}           & \multicolumn{3}{c|}{Brent} \\
  & \multicolumn{1}{|l}{R} & S & Verdict & R & S & Verdict        \\
  \hline
}

\newcommand {\testtablefooter}
{
  \hline
}

\newenvironment{testtable}
{
  \begin{tabular}{|l|l l l|l l l|}
  \testtableheader
}
{
  \testtablefooter
  \end{tabular}
}

%--
%  Shortcut command for creating graph
%--
\newcommand{\testgraphquery}[1]
{
\begin{tikzpicture}
\begin{axis}[
xlabel={Jumlah maksimal bohong $e$},
ylabel={Jumlah query},
legend pos=north west,
grid style=dashed,
legend entries={Pengulangan biner,Kode biner,Tabel pencarian},
]
\addplot[blue,mark=triangle] table{data/queryR#1.dat};
\addplot[green,mark=square] table{data/queryL#1.dat};
\end{axis}
\end{tikzpicture}
}

%--
%  Shortcut command for QED symbol. Use it while in math environment
%--
\newcommand{\eop}{\ensuremath{\blacksquare}}

% Bugs in LaTeX are damn AMAZING %
%--
%  Preventing \addvspace to throw error due to unended paragraph
%  #1: Usual parameter entered
%--
\let\oldaddvspace\addvspace
\renewcommand\addvspace[1]
{
  \par\oldaddvspace{#1}
}

%--
%  Preventing \contentsline to throw error due to unended paragraph
%  #1, #2, #3: Usual parameter entered
%--
\let\oldcontentsline\contentsline
\renewcommand\contentsline[3]
{
  \par\oldcontentsline{#1}{#2}{#3}
}

%--
%  Creates a text to denote an empty page
%--
\newcommand\emptypage
{
  \begin{center}
    [\textit{Halaman ini sengaja dikosongkan}]
  \end{center}
  \newpage
}

%--
%  Works as if applying two \clearpage, plus some text denoting
%  the page is empty
%--
\makeatletter
\def\cleardoublepage
{
  \clearpage
  \if@twoside
    \ifodd\c@page
      % do nothing
    \else
      \emptypage
    \fi
  \fi
}
\makeatother
%---------------------------------------------------------%
%--          Labelling Utilities          --%
%---------------------------------------------------------%

%----------- 1 Document hierarchies

%--
%  Auto label chapter with proper prefix
%  Param
%  #1: Label name. If not given, #2 will be used
%  #2:  The shown name
%--
\makeatletter

\let\oldchapter\chapter

\newcommand{\chapterstar}[1]
{
  \oldchapter*{#1}
  \protect\label{sec:#1}
}

\newcommand{\chapternostar}[2][]
{
  \ifstrempty{#1}
    {\oldchapter{#2}\protect\label{sec:#2}}
    {\oldchapter[#1]{#2}\protect\label{sec:#1}}
}
\renewcommand{\chapter}{\@ifstar{\chapterstar}{\chapternostar}}

\makeatother

%--
%  Auto label section with proper prefix
%  Param
%  #1: Label name. If not given, #2 will be used
%  #2:  The shown name
%--
\let\oldsection\section
\renewcommand\section[2][]
{
  \protect\oldsection{#2}
  \ifstrempty{#1}{\protect\label{sec:#2}}{\protect\label{sec:#1}}
}

%--
%  Auto label subsection with proper prefix
%  Param
%  #1: Label name. If not given, #2 will be used
%  #2:  The shown name
%--
\let\oldsubsection\subsection
\renewcommand\subsection[2][]
{
  \protect\oldsubsection{#2}
  \ifstrempty{#1}{\protect\label{sec:#2}}{\protect\label{sec:#1}}
}

%--
%  Auto label subsubsection with proper prefix
%  Param
%  #1: Label name. If not given, #2 will be used
%  #2:  The shown name
%--
\let\oldsubsubsection\subsubsection
\renewcommand\subsubsection[2][]
{
  \protect\oldsubsubsection{#2}
  \ifstrempty{#1}{\protect\label{sec:#2}}{\protect\label{sec:#1}}
}

%--
%  Auto label paragraph with proper prefix
%  Param
%  #1: Label name. If not given, #2 will be used
%  #2:  The shown name
%--
\let\oldparagraph\paragraph
\renewcommand\paragraph[2][]
{
  \protect\oldparagraph{#2}
  \ifstrempty{#1}{\protect\label{sec:#2}}{\protect\label{sec:#1}}
}

%----------- 2 Environment

%--
%  Environment for code listings. Used for proper labelling
%  Param
%  #1: Additional key-value pair
%  #2:  Caption name
%  #3: Label name, automatically prefixed
%--
\lstnewenvironment{code}[3][]
{
  \lstset{
    caption=#2,
    label=code:#3,
    #1
  }
}{}