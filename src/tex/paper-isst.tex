\documentclass{TTP_DSL2006}

%\usepackage{xparse}
\usepackage{amsmath}
\usepackage[T1]{fontenc}
\usepackage[utf8]{inputenc}
\usepackage{amssymb}
\usepackage{caption}
\usepackage{exscale}
\usepackage{fontspec} 
\usepackage{graphicx}
\usepackage{fancyvrb}
\usepackage{hyperref}
\usepackage{subfigure}
\usepackage{times}
\usepackage{xesearch}
\usepackage{tikz}
\usepackage{pgfplots}

\usetikzlibrary{backgrounds}
\usetikzlibrary{shapes.geometric, arrows}
\usetikzlibrary{automata,positioning}

% Theorems
\newtheorem{exmp}{Example}[section]

% Override fonts to Times New Roman and Arial%
\renewcommand{\large}{\fontsize{14}{18pt}\selectfont}
\renewcommand{\small}{\fontsize{11}{13.6pt}\selectfont}
\setmainfont{Times New Roman}
\setsansfont{Arial}

% New commands for quick editing of the document%
\newcommand{\titleformat}{\sffamily\bfseries \large}  %  <--- Doc. title
\newcommand{\authorformat}{\sffamily \large}            %  <--- Authors
\newcommand{\keywordsformat}{\noindent \small \sffamily}  %  <--- Kyewords
\newcommand{\abstractformat}{\noindent \textbf}              %  <--- Abstract
\newcommand{\contentformat}{\rmfamily \normalsize\vspace{18pt}}  %  <--- Main content
\newcommand{\email}{\sffamily \small \vspace{-8pt}}          %  <--- E-mail
\renewcommand{\subsection}{\textbf}

\renewcommand{\vec}{\overrightarrow}

% Make all internal and external links - black color%
\hypersetup{
    colorlinks,%
    citecolor=black,%
    filecolor=black,%
    linkcolor=black,%
    urlcolor=black
}

% Set the basic parameters of the page. DO NOT CHANGE!%
\special{papersize=210mm,297mm}
\textheight=25.6cm

% correct bad hyphenation here
\hyphenation{op-tical net-works semi-conduc-tor}

\bibliographystyle{acm}

\begin{document}

\title{\titleformat Exhaustive Search of Binary Code Generator for Solving Non\\Interactive Ulam's Searching Problem with Many Lies}

\author{\authorformat Risyanggi Azmi Faizin$^{\rm{a}}$\text{,} R. V. Hari Ginardi$^{\rm{b}}$\text{,} and Rully Soelaiman$^{\rm{c}}$}

\institute{\sffamily Institut Teknologi Sepupuh Nopember, Indonesia}

\maketitle

\begin{center}
\email{ $^{\rm a}$risyanggi@gmail.com, $^{\rm b}$hari@its.ac.id, $^{\rm c}$rully@is.its.ac.id}
\end{center}

\keywordsformat{{\textbf{Keywords:}} ulam game; lies; hamming distances; binary codes; query;}

\contentformat

\abstractformat{Abstract.}
On the classic Ulam and Rényi searching problem, the questioner has to ask some yes-no questions to find an unknown value within the agreed search space, but the answerer is allowed to lie.
There are already solutions of some variations in the Ulam and Rényi searching problem, i.e. on the type of query between range or subset and the maximum number of lies between one, two, three, and so on.
But there is no perfect solution for non-interactive queries which the answerer can only answer the questioner's query after the questioner has finished querying all the queries.
No research has resolved this problem yet.
In this paper we will describe the perfect solution for non-interactive Ulam and Rényi searching problem with many lies using greedy exhaustive search to generate binary code with Hamming distance.
The algorithm results in this paper show a much smaller number of queries than the common binary repetition algorithm.


\section{Introduction}

In the development of information technology over the last few decades, information technology is often used as a solution to the problems that ever existed, which was previously solved manually by humans. Examples of problems that ever existed were one of the classic Ulam and Rényi searching problem. This problem can be illustrated by the presence of two players called questioner and answerer. Given search space $S_M = \{0,\ldots,M-1\}$, answerer choose a value $x \in S_M$. Questioner must find the value $x$ by asking some yes-no question, is "$x \in Q$?", where $Q \subset S_M$, then answerer answer with "yes" or "no". The main problem is the answerers can lie up $e$ times. The problem with this game is to find the minimum number of queries to be able to determine the value of $x$.

On the Ulam and Rényi searching problem, questioner and answerer must agree on some rules before playing. They include search space constraints, limits on how answerers are allowed to lie, question formats, and how interactions between answerer and questioner \cite{Pelc2002}. First, questioner and answerer agree on the search space $S_M$, which has $M$ numbers, answerer can only choose a number $x$ within $\{0,\ldots,M-1\}$.

The rule of how the answerers are allowed to lie is the fundamental rule in the Ulam and Rényi search game. The rules of probability lie triggered by Rényi and the lie number rule triggered by Ulam \cite{StanislawMUlam1976}. On a globally bounded error probability, the probability of the answerer may lie is $r$, so the answerer may lie up to $rn$ times where $n$ is number of question and $r<1$ \cite{Dhagat1999}. On the rule of how many times the answerer can lie, the number of answerer can lie vary between only one \cite{Ellis2008} \cite{Pelc1988}, two \cite{Cicalese2000}, three \cite{Negro1992}, and more than three lies \cite{Berlekamp1998} \cite{Deppe2004}.

In the question format rule, there are several variations. The first is the comparative question, the question form is "Is $x<a$?" where $a \in S_M$ \cite{Innes} \cite{Auletta1992}. Then there are interval and bi-interval questions, the question form is "Is $x$ in the interval $[a, b]$?" \cite{Peter2017} and "Is $x$ in the interval $[a, b] \cup [c, d]$?" \cite{Mundici1997}. Then the subset question format, the question form is "Is $x$ in $A$ where $A \subseteq S_M$" \cite{Katona} \cite{Macula1997}.

In the interaction between questioner and answerer, there are three variations which are interactive, batch, and non-interactive. The most commonly used rule is interactive, where the answerer must answer each question asked by the questioner before the questioner asks the next question. In batch rules, the questioner and the answerer agree on how many batches. Then in each batch, the asker gives a few questions, then the answerer gives answers to the questions asked by the questioner \cite{Cicalese2000}. The last rule is non-interactive, that the answerer must answer all the questions at once \cite{Macula1997}.

Variation of Ulam and Rényi problems raised in this study is Ulam searching problem with $n$ query subset ${q_1,q_2,\ldots,q_n} | q_i \in S_M$, answerer can lie up to $e$ answers, and the query is non-interactive. No research has resolved this problem yet. Therefore this study aims to provide solutions to this problem.

There haven't been discussion of Ulam problem with non-interactive queries and more than one lie number, which the questioner must prepare all of his queries before the answerer answer all the query requests at once. The solution generated from this problem can be used to create error correctiong encoding for sending data over a noisy channel, ie when $M$ maca data is sent, there may be an error $e$ bits.

\section{Problem formulation}

The form of the Ulam problem discussed in this paper is taken from online judge SPOJ by Micha\l{} Miodek \cite{guessn5}. Answerer specifies $x$, a number in the range $S_M = \{0, \ldots, M-1 \}$. You as the questioner must find $x$ by giving the as minimal as possible $n$ query. The form of query is "is $x \in Q$?". Then the answerer answers "yes" or "no" to each query in question. The answerers can lie up to $e$ times. Additionally, the answerer may only answer the questioner's queries after the questioner has finished asking all the queries. The main problem is how to find the minimum number of queries to be able to determine the value of $x$.

The form of the query is string $s_1s_2s_3\ldots s_n$ where $s_i \in \{0,1\}$. The answer of the answerer is "Yes" if $s_x = 1$ or "No" if $s_x = 0$ assuming the answerer tell the truth. The real task of the problem is not to find value of $x$, but to just set up queries that will allow it to get value of $x$ from all possible answers from answerer. The answerer will not answer the query provided by the questioner. If the answerer find that there is a set of answers that cause more than one possible value of $x$, then testing is considered a failure.


\section{Backround of Binary Code}

\noindent \textbf{Coding Theory.}
The main purpose of coding theory is how to send a message on a noisy channel \cite{VanLint2016}. Suppose if there are eight kinds of messages to be sent, then we represent the message into binary bitstring with length 3. However if the message is sent directly through the channel containing the noise, suppose there is 1 bit will be swapped, eg $001$ to $011$, then a word can be confused into another word.

We know that if the binary code with length $n$ is used to make $2^n$ bitstring can't detect an error. The most likely idea is that the sender and receiver agree on a bitstring encryption method into a longer bitstring and can detect a maximum of $e$ errors using a linear code.

Binary code is a collection of $M$ binary bitstrings with length $n$ and Hamming distance between every two different bitstrings are maximum $d$. Example $M=8$, $n=6$, and $d=3$ in Fig.~\ref{fig:binarycode683}. Parameter in this binary code is $(6,8,3)_2$, binary is shown by number 2, bitstring length 6, contains 8 bitstrings, and minimum Hamming distance 3. Bitstring in binary code hereinafter called codeword. General notation of binary code is $(n,M,d)_2$.

\begin{figure}
\centering
\begin{BVerbatim}
000000  100110
001011  101101
010101  110011
011110  111000
\end{BVerbatim}
\caption{Binary code $(6,8,3)_2$}
\label{fig:binarycode683}
\end{figure}

% With binary code $(6,8,3)_2$ in Fig.~\ref{fig:binarycode683}, sender and receiver agreed on only those codeword that will be sent and received. Assumed only one bit will be error, the message can still be recovered to the original message. Hamming distance between two different codewords is 3, that means in Jarak Hamming antara setiap dua kata kode yang berbeda adalah 3, berarti dari setiap kata kode, terdapat sejumlah bitstring selain kata kode berjarak 1.

% We can assume there are $M$ balls with radius $e$ that are not intersecting each others. Distance between one ball to another is at least $d$, so that we can obtain Eq.~\ref{eq:de}.

\begin{equation} \label{eq:de}
d = 2e + 1
\end{equation}

\noindent \textbf{Generate Binary Code.}
An $(n,M,d)_2$ binary code can be generated by using linear combination of each row of a generator matrix $[n,m,d]_2$ where $2^m = M$. A Generator matrix $G$ is an $n \times m$ matrix. Now see that binary code $(6,8,3)_2$ in Figure \ref{fig:binarycode683} can be generated using generator matrix $[6,3,3]_2$ in Figure \ref{fig:generator633}. Assume that each row of $[6,3,3]_2$ is $g_1$, $g_2$, and $g_3$, then binary code $(6,8,3)_2$ is the linear combination of all vectors of the form ${\lambda}_1 g_1 + {\lambda}_2 g_2 + {\lambda}_3 g_3$ where $\lambda{_i} \in \mathbb{F}_2^6$.

\begin{figure}
\centering
\begin{BVerbatim}
100110
010101
001011
\end{BVerbatim}
\caption{Generator matrix $[6,3,3]_2$}
\label{fig:generator633}
\end{figure}

From a binary code, we can create a new binary code by modifying the existing binary code \cite{Huffman}. Modifications that can be done is to add and reduce the code. From a binary code $(n,M,d)_2 $, it can be deducted on a particular column so the binary code becomes $(n-1,M,d^\prime) _2 $ where $d^\prime = d$ or $d^\prime = d-1$. Similarly with the addition of code, from a binary code $(n,M,d)_2$, it can be added code so the binary code becomes $(n+1,M,d^\prime)_2 $ where $d^\prime = d$ or $d^\prime = d+1$.

\section{Greedy Exhaustive Search}

We know that the Ulam problem with the $M$ search space and the maximum $e$ lies can be solved by making $q$ queries which if the set of the queries is transposed, it will form the binary code $(n,M,d) _2$ where $d=2*e+1$. Therefore the next stage is how to find as minimum as possible $n$ to make the binary code $(n,M,d)_2$ where $M$ and $d$ is known. The main step to make the binary code with $M$ and $e$ known is to add the code iteratively until its minimum Hamming distance reaches $e$.

The solution using the matrix generator should make a certain $s$ sequence such that a minimum number of $n$ queries generated from $\vec{s}= \{s_1,\ldots,s_n\}$ will generate as many as possible Hamming distance $d$. For that reason, a greedy exhaustive search is proposed to get the most optimized $\vec{s}$ sequences. The greedy approach is used because the global optimum sought is as many as possible the minimum distance value with as few queries as possible. Global optimum can be obtained by finding local optimum in each iteration.

In the greedy algorithm, there are four kinds of selection functions to select the best query candidate on each iteration:
\begin{enumerate}
  \item The most minimum distance;
  \item The most minimum distance, then the least maximum distance;
  \item The most {maximum distance - minimum distance};
  \item The most minimum distance, then the least min count distance.
\end{enumerate}

The four selection functions are determined because the global optimum sought is the largest minimum distance, the difference between the maximum distance and minimum distances should be as small as possible, and the resulting query can eliminate as much as possible the number with minimum distance. The four selection functions will be tested to choose the best selection function and used in the main greedy exhaustive search algorithm.

\section{Testing and Evaluation.}

The testing phase is to test the dataset according to the limit of the problem to find out the results and the performance of the algorithm. Testing will also be used to compare the number of queries, execution time, and memory of the algorithm from all dataset test cases. The dataset will be varied according to the $2 \le M \le 4096$ search space and the maximal $2 \le e \le 16$ lies.

In testing four gredy selection functions, the result shows the number of queries from each selection function, the less the query in all test cases the better. The greedy algorithm with four variations of the selection function will be compared with the bin truth binary repetition algorithm. The result of binary repetition algorithm will also be the maximum query limit, if the number of queries resulted in greedy algorithm exceeds the number of query result of binary repetition, then the number of binary repetition query to be selected.

\begin{figure}
	\centering
	\begin{tikzpicture}
	\begin{axis}[
	xlabel={Maximal lies $e$},
	ylabel={Number of queries},
	legend pos=north west,
	grid style=dashed,
	legend entries={1,2,3,4,5},
	]
	\addplot[green,mark=*] table{data/brute-min.dat};
	\addplot[red,mark=o] table{data/brute-min-max.dat};
	\addplot[blue,mark=o] table{data/brute-max-min.dat};
	\addplot[purple,mark=*] table{data/brute-min-mincount.dat};
	\addplot[black,mark=o] table{data/brute-repetition.dat};
	\end{axis}
	\end{tikzpicture}
	\caption{Testing of selection function greedy algorithm}
	\label{fig:graph_selection_function}
\end{figure}

\begin{figure}
	\centering
	\begin{tikzpicture}
	\begin{axis}[
	xlabel={Search space ($2^m$)},
	ylabel={Execution time (sec)},
	legend pos=north west,
	grid style=dashed,
	legend entries={Binary repetition,Binary code,Lookup table},
	]
	\addplot[blue,mark=triangle] table{data/timeR.dat};
	\addplot[red,mark=*] table{data/timeB.dat};
	\addplot[green,mark=square] table{data/timeL.dat};
	\end{axis}
	\end{tikzpicture}
	\caption{Execution time comparation graph}
	\label{fig:graph_time}
\end{figure}

\begin{figure}
	\centering
	\begin{tikzpicture}
	\begin{axis}[
	xlabel={Search space ($2^m$)},
	ylabel={Memory usage (byte)},
	legend pos=north west,
	grid style=dashed,
	legend entries={Binary repetition,Binary code,Lookup table},
	]
	\addplot[blue,mark=triangle] table{data/memoryR.dat};
	\addplot[red,mark=*] table{data/memoryB.dat};
	\addplot[green,mark=square] table{data/memoryL.dat};
	\end{axis}
	\end{tikzpicture}
	\caption{Memory usage comparation graph}
	\label{fig:graph_memory}
\end{figure}

In Fig.~\ref{fig:graph_selection_function}, legends number one to five are respectively the largest minimum distance, the largest minimum distance then the smallest maximum distance, the maximum distance reducted with the smallest minimum distance, the largest minimum distance then the number of numbers has the smallest minimum value, and last is binary repetition algorithm. From the figure it can be concluded that the most optimal selection function is the largest minimum followed by the number of numbers has the smallest minimum value and the selection function is used in the main greedy exhaustive search algorithm.

In the test of execution time and memory usage, on each algorithm, tested 12 test cases according to the number of search space variations $M = 2^m \mid 1 \leq m \leq 12$. In each test case, there is 15 test cases according to the maximum lie variation $2 \leq e \leq 16$. The test results are shown in Fig.~\ref{fig:graph_time} and Fig.~\ref{fig:graph_memory}. It can be concluded that binary code algorithm without search table resulted in a higher execution time and memory usage, while for binary repetition and binary code with lookup table result in stable execution time and memory usage. This is because the binary code algorithm without the lookup table has the complexity of $O(M^3) $ in the exhaustive search process, whereas the binary code algorithm with the lookup table has the $O(1)$ complexity because the exhaustive search process is done as a pre-process.

\section{Conclusion}
\noindent The conclusion is that the generating binary code with greedy exhaustive search algorithm with greedy selection function used is the largest minimum followed by the number of numbers has the smallest minimum value shows much better results in terms of the number of queries produced, but the execution time and memory usage. Therefore, binary code algorithm can be optimized with lookup table as preprocess so that it can reduce and stabilize the execution time.

\bibliography{holy}

\end{document}